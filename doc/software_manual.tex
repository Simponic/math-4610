% Created 2023-11-10 Fri 20:54
% Intended LaTeX compiler: pdflatex
\documentclass[11pt]{article}
\usepackage[utf8]{inputenc}
\usepackage[T1]{fontenc}
\usepackage{graphicx}
\usepackage{longtable}
\usepackage{wrapfig}
\usepackage{rotating}
\usepackage[normalem]{ulem}
\usepackage{amsmath}
\usepackage{amssymb}
\usepackage{capt-of}
\usepackage{hyperref}
\notindent \notag  \usepackage{amsmath} \usepackage[a4paper,margin=1in,portrait]{geometry}
\author{Elizabeth Hunt}
\date{\today}
\title{LIZFCM Software Manual (v0.3)}
\hypersetup{
 pdfauthor={Elizabeth Hunt},
 pdftitle={LIZFCM Software Manual (v0.3)},
 pdfkeywords={},
 pdfsubject={},
 pdfcreator={Emacs 29.1 (Org mode 9.7-pre)}, 
 pdflang={English}}
\begin{document}

\maketitle
\tableofcontents

\setlength\parindent{0pt}
\section{Design}
\label{sec:orgdac8577}
The LIZFCM static library (at \url{https://github.com/Simponic/math-4610}) is a successor to my
attempt at writing codes for the Fundamentals of Computational Mathematics course in Common
Lisp, but the effort required to meet the requirement of creating a static library became
too difficult to integrate outside of the \texttt{ASDF} solution that Common Lisp already brings
to the table.

All of the work established in \texttt{deprecated-cl} has been painstakingly translated into
the C programming language. I have a couple tenets for its design:

\begin{itemize}
\item Implementations of routines should all be done immutably in respect to arguments.
\item Functional programming is good (it's\ldots{} rough in C though).
\item Routines are separated into "modules" that follow a form of separation of concerns
in files, and not individual files per function.
\end{itemize}
\section{Compilation}
\label{sec:org7755023}
A provided \texttt{Makefile} is added for convencience. It has been tested on an \texttt{arm}-based M1 machine running
MacOS as well as \texttt{x86} Arch Linux.

\begin{enumerate}
\item \texttt{cd} into the root of the repo
\item \texttt{make}
\end{enumerate}

Then, as of homework 5, the testing routines are provided in \texttt{test} and utilize the
\texttt{utest} "micro"library. They compile to a binary in \texttt{./dist/lizfcm.test}.

Execution of the Makefile will perform compilation of individual routines.

But, in the requirement of manual intervention (should the little alien workers
inside the computer fail to do their job), one can use the following command to
produce an object file:

\begin{verbatim}
  gcc -Iinc/ -lm -Wall -c src/<the_routine>.c -o build/<the_routine>.o
\end{verbatim}

Which is then bundled into a static library in \texttt{lib/lizfcm.a} and can be linked
in the standard method.
\section{The LIZFCM API}
\label{sec:org940357c}
\subsection{Simple Routines}
\label{sec:org28486b0}
\subsubsection{\texttt{smaceps}}
\label{sec:org1de3a4e}
\begin{itemize}
\item Author: Elizabeth Hunt
\item Name: \texttt{smaceps}
\item Location: \texttt{src/maceps.c}
\item Input: none
\item Output: a \texttt{float} returning the specific "Machine Epsilon" of a machine on a
single precision floating point number at which it becomes "indistinguishable".
\end{itemize}

\begin{verbatim}
float smaceps() {
  float one = 1.0;
  float machine_epsilon = 1.0;
  float one_approx = one + machine_epsilon;

  while (fabsf(one_approx - one) > 0) {
    machine_epsilon /= 2;
    one_approx = one + machine_epsilon;
  }

  return machine_epsilon;
}
\end{verbatim}
\subsubsection{\texttt{dmaceps}}
\label{sec:org742e61e}
\begin{itemize}
\item Author: Elizabeth Hunt
\item Name: \texttt{dmaceps}
\item Location: \texttt{src/maceps.c}
\item Input: none
\item Output: a \texttt{double} returning the specific "Machine Epsilon" of a machine on a
double precision floating point number at which it becomes "indistinguishable".
\end{itemize}

\begin{verbatim}
double dmaceps() {
  double one = 1.0;
  double machine_epsilon = 1.0;
  double one_approx = one + machine_epsilon;

  while (fabs(one_approx - one) > 0) {
    machine_epsilon /= 2;
    one_approx = one + machine_epsilon;
  }

  return machine_epsilon;
}
\end{verbatim}
\subsection{Derivative Routines}
\label{sec:org21233d3}
\subsubsection{\texttt{central\_derivative\_at}}
\label{sec:org6a00f6c}
\begin{itemize}
\item Author: Elizabeth Hunt
\item Name: \texttt{central\_derivative\_at}
\item Location: \texttt{src/approx\_derivative.c}
\item Input:
\begin{itemize}
\item \texttt{f} is a pointer to a one-ary function that takes a double as input and produces
a double as output
\item \texttt{a} is the domain value at which we approximate \texttt{f'}
\item \texttt{h} is the step size
\end{itemize}
\item Output: a \texttt{double} of the approximate value of \texttt{f'(a)} via the central difference
method.
\end{itemize}

\begin{verbatim}
double central_derivative_at(double (*f)(double), double a, double h) {
  assert(h > 0);

  double x2 = a + h;
  double x1 = a - h;

  double y2 = (*f)(x2);
  double y1 = (*f)(x1);

  return (y2 - y1) / (x2 - x1);
}
\end{verbatim}
\subsubsection{\texttt{forward\_derivative\_at}}
\label{sec:org78f40a9}
\begin{itemize}
\item Author: Elizabeth Hunt
\item Name: \texttt{forward\_derivative\_at}
\item Location: \texttt{src/approx\_derivative.c}
\item Input:
\begin{itemize}
\item \texttt{f} is a pointer to a one-ary function that takes a double as input and produces
a double as output
\item \texttt{a} is the domain value at which we approximate \texttt{f'}
\item \texttt{h} is the step size
\end{itemize}
\item Output: a \texttt{double} of the approximate value of \texttt{f'(a)} via the forward difference
method.
\end{itemize}

\begin{verbatim}
double forward_derivative_at(double (*f)(double), double a, double h) {
  assert(h > 0);

  double x2 = a + h;
  double x1 = a;

  double y2 = (*f)(x2);
  double y1 = (*f)(x1);

  return (y2 - y1) / (x2 - x1);
}
\end{verbatim}
\subsubsection{\texttt{backward\_derivative\_at}}
\label{sec:org888d29e}
\begin{itemize}
\item Author: Elizabeth Hunt
\item Name: \texttt{backward\_derivative\_at}
\item Location: \texttt{src/approx\_derivative.c}
\item Input:
\begin{itemize}
\item \texttt{f} is a pointer to a one-ary function that takes a double as input and produces
a double as output
\item \texttt{a} is the domain value at which we approximate \texttt{f'}
\item \texttt{h} is the step size
\end{itemize}
\item Output: a \texttt{double} of the approximate value of \texttt{f'(a)} via the backward difference
method.
\end{itemize}

\begin{verbatim}
double backward_derivative_at(double (*f)(double), double a, double h) {
  assert(h > 0);

  double x2 = a;
  double x1 = a - h;

  double y2 = (*f)(x2);
  double y1 = (*f)(x1);

  return (y2 - y1) / (x2 - x1);
}
\end{verbatim}
\subsection{Vector Routines}
\label{sec:org73b87ea}
\subsubsection{Vector Arithmetic: \texttt{add\_v, minus\_v}}
\label{sec:orgf8b5da1}
\begin{itemize}
\item Author: Elizabeth Hunt
\item Name(s): \texttt{add\_v}, \texttt{minus\_v}
\item Location: \texttt{src/vector.c}
\item Input: two pointers to locations in memory wherein \texttt{Array\_double}'s lie
\item Output: a pointer to a new \texttt{Array\_double} as the result of addition or subtraction
of the two input \texttt{Array\_double}'s
\end{itemize}

\begin{verbatim}
Array_double *add_v(Array_double *v1, Array_double *v2) {
  assert(v1->size == v2->size);

  Array_double *sum = copy_vector(v1);
  for (size_t i = 0; i < v1->size; i++)
    sum->data[i] += v2->data[i];
  return sum;
}

Array_double *minus_v(Array_double *v1, Array_double *v2) {
  assert(v1->size == v2->size);

  Array_double *sub = InitArrayWithSize(double, v1->size, 0);
  for (size_t i = 0; i < v1->size; i++)
    sub->data[i] = v1->data[i] - v2->data[i];
  return sub;
}
\end{verbatim}
\subsubsection{Norms: \texttt{l1\_norm}, \texttt{l2\_norm}, \texttt{linf\_norm}}
\label{sec:orgc5368a1}
\begin{itemize}
\item Author: Elizabeth Hunt
\item Name(s): \texttt{l1\_norm}, \texttt{l2\_norm}, \texttt{linf\_norm}
\item Location: \texttt{src/vector.c}
\item Input: a pointer to a location in memory wherein an \texttt{Array\_double} lies
\item Output: a \texttt{double} representing the value of the norm the function applies
\end{itemize}

\begin{verbatim}
double l1_norm(Array_double *v) {
  double sum = 0;
  for (size_t i = 0; i < v->size; ++i)
    sum += fabs(v->data[i]);
  return sum;
}

double l2_norm(Array_double *v) {
  double norm = 0;
  for (size_t i = 0; i < v->size; ++i)
    norm += v->data[i] * v->data[i];
  return sqrt(norm);
}

double linf_norm(Array_double *v) {
  assert(v->size > 0);
  double max = v->data[0];
  for (size_t i = 0; i < v->size; ++i)
    max = c_max(v->data[i], max);
  return max;
}
\end{verbatim}
\subsubsection{\texttt{vector\_distance}}
\label{sec:org0505e0b}
\begin{itemize}
\item Author: Elizabeth Hunt
\item Name: \texttt{vector\_distance}
\item Location: \texttt{src/vector.c}
\item Input: two pointers to locations in memory wherein \texttt{Array\_double}'s lie, and a pointer to a
one-ary function \texttt{norm} taking as input a pointer to an \texttt{Array\_double} and returning a double
representing the norm of that \texttt{Array\_double}
\end{itemize}

\begin{verbatim}
double vector_distance(Array_double *v1, Array_double *v2,
                       double (*norm)(Array_double *)) {
  Array_double *minus = minus_v(v1, v2);
  double dist = (*norm)(minus);
  free(minus);
  return dist;
}
\end{verbatim}
\subsubsection{Distances: \texttt{l1\_distance}, \texttt{l2\_distance}, \texttt{linf\_distance}}
\label{sec:org1c45dae}
\begin{itemize}
\item Author: Elizabeth Hunt
\item Name(s): \texttt{l1\_distance}, \texttt{l2\_distance}, \texttt{linf\_distance}
\item Location: \texttt{src/vector.c}
\item Input: two pointers to locations in memory wherein \texttt{Array\_double}'s lie, and the distance
via the corresponding \texttt{l1}, \texttt{l2}, or \texttt{linf} norms
\item Output: A \texttt{double} representing the distance between the two \texttt{Array\_doubles}'s by the given
norm.
\end{itemize}

\begin{verbatim}
double l1_distance(Array_double *v1, Array_double *v2) {
  return vector_distance(v1, v2, &l1_norm);
}

double l2_distance(Array_double *v1, Array_double *v2) {
  return vector_distance(v1, v2, &l2_norm);
}

double linf_distance(Array_double *v1, Array_double *v2) {
  return vector_distance(v1, v2, &linf_norm);
}
\end{verbatim}
\subsubsection{\texttt{sum\_v}}
\label{sec:org687d1bd}
\begin{itemize}
\item Author: Elizabeth Hunt
\item Name: \texttt{sum\_v}
\item Location: \texttt{src/vector.c}
\item Input: a pointer to an \texttt{Array\_double}
\item Output: a \texttt{double} representing the sum of all the elements of an \texttt{Array\_double}
\end{itemize}

\begin{verbatim}
double sum_v(Array_double *v) {
  double sum = 0;
  for (size_t i = 0; i < v->size; i++)
    sum += v->data[i];
  return sum;
}
\end{verbatim}
\subsubsection{\texttt{scale\_v}}
\label{sec:org5926df1}
\begin{itemize}
\item Author: Elizabeth Hunt
\item Name: \texttt{scale\_v}
\item Location: \texttt{src/vector.c}
\item Input: a pointer to an \texttt{Array\_double} and a scalar \texttt{double} to scale the vector
\item Output: a pointer to a new \texttt{Array\_double} of the scaled input \texttt{Array\_double}
\end{itemize}

\begin{verbatim}
Array_double *scale_v(Array_double *v, double m) {
  Array_double *copy = copy_vector(v);
  for (size_t i = 0; i < v->size; i++)
    copy->data[i] *= m;
  return copy;
}
\end{verbatim}
\subsubsection{\texttt{free\_vector}}
\label{sec:org3458f6a}
\begin{itemize}
\item Author: Elizabeth Hunt
\item Name: \texttt{free\_vector}
\item Location: \texttt{src/vector.c}
\item Input: a pointer to an \texttt{Array\_double}
\item Output: nothing.
\item Side effect: free the memory of the reserved \texttt{Array\_double} on the heap
\end{itemize}

\begin{verbatim}
void free_vector(Array_double *v) {
  free(v->data);
  free(v);
}
\end{verbatim}
\subsubsection{\texttt{add\_element}}
\label{sec:org54cba50}
\begin{itemize}
\item Author: Elizabeth Hunt
\item Name: \texttt{add\_element}
\item Location: \texttt{src/vector.c}
\item Input: a pointer to an \texttt{Array\_double}
\item Output: a new \texttt{Array\_double} with element \texttt{x} appended.
\end{itemize}

\begin{verbatim}
Array_double *add_element(Array_double *v, double x) {
  Array_double *pushed = InitArrayWithSize(double, v->size + 1, 0.0);
  for (size_t i = 0; i < v->size; ++i)
    pushed->data[i] = v->data[i];
  pushed->data[v->size] = x;
  return pushed;
}
\end{verbatim}
\subsubsection{\texttt{slice\_element}}
\label{sec:org02cd40a}
\begin{itemize}
\item Author: Elizabeth Hunt
\item Name: \texttt{slice\_element}
\item Location: \texttt{src/vector.c}
\item Input: a pointer to an \texttt{Array\_double}
\item Output: a new \texttt{Array\_double} with element \texttt{x} sliced.
\end{itemize}

\begin{verbatim}
Array_double *slice_element(Array_double *v, size_t x) {
  Array_double *sliced = InitArrayWithSize(double, v->size - 1, 0.0);
  for (size_t i = 0; i < v->size - 1; ++i)
    sliced->data[i] = i >= x ? v->data[i + 1] : v->data[i];
  return sliced;
}
\end{verbatim}
\subsubsection{\texttt{copy\_vector}}
\label{sec:org4b0c599}
\begin{itemize}
\item Author: Elizabeth Hunt
\item Name: \texttt{copy\_vector}
\item Location: \texttt{src/vector.c}
\item Input: a pointer to an \texttt{Array\_double}
\item Output: a pointer to a new \texttt{Array\_double} whose \texttt{data} and \texttt{size} are copied from the input
\texttt{Array\_double}
\end{itemize}

\begin{verbatim}
Array_double *copy_vector(Array_double *v) {
  Array_double *copy = InitArrayWithSize(double, v->size, 0.0);
  for (size_t i = 0; i < copy->size; ++i)
    copy->data[i] = v->data[i];
  return copy;
}
\end{verbatim}
\subsubsection{\texttt{format\_vector\_into}}
\label{sec:orgde12441}
\begin{itemize}
\item Author: Elizabeth Hunt
\item Name: \texttt{format\_vector\_into}
\item Location: \texttt{src/vector.c}
\item Input: a pointer to an \texttt{Array\_double} and a pointer to a c-string \texttt{s} to "print" the vector out
into
\item Output: nothing.
\item Side effect: overwritten memory into \texttt{s}
\end{itemize}

\begin{verbatim}
void format_vector_into(Array_double *v, char *s) {
  if (v->size == 0) {
    strcat(s, "empty");
    return;
  }

  for (size_t i = 0; i < v->size; ++i) {
    char num[64];
    strcpy(num, "");

    sprintf(num, "%f,", v->data[i]);
    strcat(s, num);
  }
  strcat(s, "\n");
}
\end{verbatim}
\subsection{Matrix Routines}
\label{sec:orgd85d8ec}
\subsubsection{\texttt{lu\_decomp}}
\label{sec:org6a14cbd}
\begin{itemize}
\item Author: Elizabeth Hunt
\item Name: \texttt{lu\_decomp}
\item Location: \texttt{src/matrix.c}
\item Input: a pointer to a \texttt{Matrix\_double} \(m\) to decompose into a lower triangular and upper triangular
matrix \(L\), \(U\), respectively such that \(LU = m\).
\item Output: a pointer to the location in memory in which two \texttt{Matrix\_double}'s reside: the first
representing \(L\), the second, \(U\).
\item Errors: Fails assertions when encountering a matrix that cannot be
decomposed
\end{itemize}

\begin{verbatim}
Matrix_double **lu_decomp(Matrix_double *m) {
  assert(m->cols == m->rows);

  Matrix_double *u = copy_matrix(m);
  Matrix_double *l_empt = InitMatrixWithSize(double, m->rows, m->cols, 0.0);
  Matrix_double *l = put_identity_diagonal(l_empt);
  free_matrix(l_empt);

  Matrix_double **u_l = malloc(sizeof(Matrix_double *) * 2);

  for (size_t y = 0; y < m->rows; y++) {
    if (u->data[y]->data[y] == 0) {
      printf("ERROR: a pivot is zero in given matrix\n");
      assert(false);
    }
  }

  if (u && l) {
    for (size_t x = 0; x < m->cols; x++) {
      for (size_t y = x + 1; y < m->rows; y++) {
        double denom = u->data[x]->data[x];

        if (denom == 0) {
          printf("ERROR: non-factorable matrix\n");
          assert(false);
        }

        double factor = -(u->data[y]->data[x] / denom);

        Array_double *scaled = scale_v(u->data[x], factor);
        Array_double *added = add_v(scaled, u->data[y]);
        free_vector(scaled);
        free_vector(u->data[y]);

        u->data[y] = added;
        l->data[y]->data[x] = -factor;
      }
    }
  }

  u_l[0] = u;
  u_l[1] = l;
  return u_l;
}
\end{verbatim}
\subsubsection{\texttt{bsubst}}
\label{sec:org8b51171}
\begin{itemize}
\item Author: Elizabeth Hunt
\item Name: \texttt{bsubst}
\item Location: \texttt{src/matrix.c}
\item Input: a pointer to an upper-triangular \texttt{Matrix\_double} \(u\) and a \texttt{Array\_double}
\(b\)
\item Output: a pointer to a new \texttt{Array\_double} whose entries are given by performing
back substitution
\end{itemize}

\begin{verbatim}
Array_double *bsubst(Matrix_double *u, Array_double *b) {
  assert(u->rows == b->size && u->cols == u->rows);

  Array_double *x = copy_vector(b);
  for (int64_t row = b->size - 1; row >= 0; row--) {
    for (size_t col = b->size - 1; col > row; col--)
      x->data[row] -= x->data[col] * u->data[row]->data[col];
    x->data[row] /= u->data[row]->data[row];
  }
  return x;
}
\end{verbatim}
\subsubsection{\texttt{fsubst}}
\label{sec:orgf9180a0}
\begin{itemize}
\item Author: Elizabeth Hunt
\item Name: \texttt{fsubst}
\item Location: \texttt{src/matrix.c}
\item Input: a pointer to a lower-triangular \texttt{Matrix\_double} \(l\) and a \texttt{Array\_double}
\(b\)
\item Output: a pointer to a new \texttt{Array\_double} whose entries are given by performing
forward substitution
\end{itemize}

\begin{verbatim}
Array_double *fsubst(Matrix_double *l, Array_double *b) {
  assert(l->rows == b->size && l->cols == l->rows);

  Array_double *x = copy_vector(b);

  for (size_t row = 0; row < b->size; row++) {
    for (size_t col = 0; col < row; col++)
      x->data[row] -= x->data[col] * l->data[row]->data[col];
    x->data[row] /= l->data[row]->data[row];
  }

  return x;
}
\end{verbatim}
\subsubsection{\texttt{solve\_matrix\_lu\_bsubst}}
\label{sec:orgf3845f4}
\begin{itemize}
\item Author: Elizabeth Hunt
\item Location: \texttt{src/matrix.c}
\item Input: a pointer to a \texttt{Matrix\_double} \(m\) and a pointer to an \texttt{Array\_double} \(b\)
\item Output: \(x\) such that \(mx = b\) if such a solution exists (else it's non LU-factorable as discussed
above)
\end{itemize}

Here we make use of forward substitution to first solve \(Ly = b\) given \(L\) as the \(L\) factor in
\texttt{lu\_decomp}. Then we use back substitution to solve \(Ux = y\) for \(x\) similarly given \(U\).

Then, \(LUx = b\), thus \(x\) is a solution.

\begin{verbatim}
Array_double *solve_matrix_lu_bsubst(Matrix_double *m, Array_double *b) {
  assert(b->size == m->rows);
  assert(m->rows == m->cols);

  Array_double *x = copy_vector(b);
  Matrix_double **u_l = lu_decomp(m);
  Matrix_double *u = u_l[0];
  Matrix_double *l = u_l[1];

  Array_double *b_fsub = fsubst(l, b);
  x = bsubst(u, b_fsub);
  free_vector(b_fsub);

  free_matrix(u);
  free_matrix(l);
  free(u_l);

  return x;
}
\end{verbatim}
\subsubsection{\texttt{gaussian\_elimination}}
\label{sec:orge926b79}
\begin{itemize}
\item Author: Elizabeth Hunt
\item Location: \texttt{src/matrix.c}
\item Input: a pointer to a \texttt{Matrix\_double} \(m\)
\item Output: a pointer to a copy of \(m\) in reduced echelon form
\end{itemize}

This works by finding the row with a maximum value in the column \(k\). Then, it uses that as a pivot, and
applying reduction to all other rows. The general idea is available at \url{https://en.wikipedia.org/wiki/Gaussian\_elimination}.

\begin{verbatim}
Matrix_double *gaussian_elimination(Matrix_double *m) {
  uint64_t h = 0;
  uint64_t k = 0;

  Matrix_double *m_cp = copy_matrix(m);

  while (h < m_cp->rows && k < m_cp->cols) {
    uint64_t max_row = 0;
    double total_max = 0.0;

    for (uint64_t row = h; row < m_cp->rows; row++) {
      double this_max = c_max(fabs(m_cp->data[row]->data[k]), total_max);
      if (c_max(this_max, total_max) == this_max) {
        max_row = row;
      }
    }

    if (max_row == 0) {
      k++;
      continue;
    }

    Array_double *swp = m_cp->data[max_row];
    m_cp->data[max_row] = m_cp->data[h];
    m_cp->data[h] = swp;

    for (uint64_t row = h + 1; row < m_cp->rows; row++) {
      double factor = m_cp->data[row]->data[k] / m_cp->data[h]->data[k];
      m_cp->data[row]->data[k] = 0.0;

      for (uint64_t col = k + 1; col < m_cp->cols; col++) {
        m_cp->data[row]->data[col] -= m_cp->data[h]->data[col] * factor;
      }
    }

    h++;
    k++;
  }

  return m_cp;
}
\end{verbatim}
\subsubsection{\texttt{solve\_matrix\_gaussian}}
\label{sec:orgc4f0d99}
\begin{itemize}
\item Author: Elizabeth Hunt
\item Location: \texttt{src/matrix.c}
\item Input: a pointer to a \texttt{Matrix\_double} \(m\) and a target \texttt{Array\_double} \(b\)
\item Output: a pointer to a vector \(x\) being the solution to the equation \(mx = b\)
\end{itemize}

We first perform \texttt{gaussian\_elimination} after augmenting \(m\) and \(b\). Then, as \(m\) is in reduced echelon form, it's an upper
triangular matrix, so we can perform back substitution to compute \(x\).

\begin{verbatim}
Array_double *solve_matrix_gaussian(Matrix_double *m, Array_double *b) {
  assert(b->size == m->rows);
  assert(m->rows == m->cols);

  Matrix_double *m_augment_b = add_column(m, b);
  Matrix_double *eliminated = gaussian_elimination(m_augment_b);

  Array_double *b_gauss = col_v(eliminated, m->cols);
  Matrix_double *u = slice_column(eliminated, m->rows);

  Array_double *solution = bsubst(u, b_gauss);

  free_matrix(m_augment_b);
  free_matrix(eliminated);
  free_matrix(u);
  free_vector(b_gauss);

  return solution;
}
\end{verbatim}
\subsubsection{\texttt{m\_dot\_v}}
\label{sec:orgb7015af}
\begin{itemize}
\item Author: Elizabeth Hunt
\item Location: \texttt{src/matrix.c}
\item Input: a pointer to a \texttt{Matrix\_double} \(m\) and \texttt{Array\_double} \(v\)
\item Output: the dot product \(mv\) as an \texttt{Array\_double}
\end{itemize}

\begin{verbatim}
Array_double *m_dot_v(Matrix_double *m, Array_double *v) {
  assert(v->size == m->cols);

  Array_double *product = copy_vector(v);

  for (size_t row = 0; row < v->size; ++row)
    product->data[row] = v_dot_v(m->data[row], v);

  return product;
}
\end{verbatim}
\subsubsection{\texttt{put\_identity\_diagonal}}
\label{sec:orge955396}
\begin{itemize}
\item Author: Elizabeth Hunt
\item Location: \texttt{src/matrix.c}
\item Input: a pointer to a \texttt{Matrix\_double}
\item Output: a pointer to a copy to \texttt{Matrix\_double} whose diagonal is full of 1's
\end{itemize}

\begin{verbatim}
Matrix_double *put_identity_diagonal(Matrix_double *m) {
  assert(m->rows == m->cols);
  Matrix_double *copy = copy_matrix(m);
  for (size_t y = 0; y < m->rows; ++y)
    copy->data[y]->data[y] = 1.0;
  return copy;
}
\end{verbatim}
\subsubsection{\texttt{slice\_column}}
\label{sec:org886997f}
\begin{itemize}
\item Author: Elizabeth Hunt
\item Location: \texttt{src/matrix.c}
\item Input: a pointer to a \texttt{Matrix\_double}
\item Output: a pointer to a copy of the given \texttt{Matrix\_double} with column at \texttt{x} sliced
\end{itemize}

\begin{verbatim}
Matrix_double *slice_column(Matrix_double *m, size_t x) {
  Matrix_double *sliced = copy_matrix(m);

  for (size_t row = 0; row < m->rows; row++) {
    Array_double *old_row = sliced->data[row];
    sliced->data[row] = slice_element(old_row, x);
    free_vector(old_row);
  }
  sliced->cols--;

  return sliced;
}
\end{verbatim}
\subsubsection{\texttt{add\_column}}
\label{sec:org405e1c5}
\begin{itemize}
\item Author: Elizabet Hunt
\item Location: \texttt{src/matrix.c}
\item Input: a pointer to a \texttt{Matrix\_double} and a new vector representing the appended column \texttt{x}
\item Output: a pointer to a copy of the given \texttt{Matrix\_double} with a new column \texttt{x}
\end{itemize}

\begin{verbatim}
Matrix_double *add_column(Matrix_double *m, Array_double *v) {
  Matrix_double *pushed = copy_matrix(m);

  for (size_t row = 0; row < m->rows; row++) {
    Array_double *old_row = pushed->data[row];
    pushed->data[row] = add_element(old_row, v->data[row]);
    free_vector(old_row);
  }

  pushed->cols++;
  return pushed;
}
\end{verbatim}
\subsubsection{\texttt{copy\_matrix}}
\label{sec:org01ea984}
\begin{itemize}
\item Author: Elizabeth Hunt
\item Location: \texttt{src/matrix.c}
\item Input: a pointer to a \texttt{Matrix\_double}
\item Output: a pointer to a copy of the given \texttt{Matrix\_double}
\end{itemize}

\begin{verbatim}
Matrix_double *copy_matrix(Matrix_double *m) {
  Matrix_double *copy = InitMatrixWithSize(double, m->rows, m->cols, 0.0);
  for (size_t y = 0; y < copy->rows; y++) {
    free_vector(copy->data[y]);
    copy->data[y] = copy_vector(m->data[y]);
  }
  return copy;
}
\end{verbatim}
\subsubsection{\texttt{free\_matrix}}
\label{sec:orgab8c2cf}
\begin{itemize}
\item Author: Elizabeth Hunt
\item Location: \texttt{src/matrix.c}
\item Input: a pointer to a \texttt{Matrix\_double}
\item Output: none.
\item Side Effects: frees memory reserved by a given \texttt{Matrix\_double} and its member
\texttt{Array\_double} vectors describing its rows.
\end{itemize}

\begin{verbatim}
void free_matrix(Matrix_double *m) {
  for (size_t y = 0; y < m->rows; ++y)
    free_vector(m->data[y]);
  free(m);
}
\end{verbatim}
\subsubsection{\texttt{format\_matrix\_into}}
\label{sec:org9e01978}
\begin{itemize}
\item Author: Elizabeth Hunt
\item Name: \texttt{format\_matrix\_into}
\item Location: \texttt{src/matrix.c}
\item Input: a pointer to a \texttt{Matrix\_double} and a pointer to a c-string \texttt{s} to "print" the vector out
into
\item Output: nothing.
\item Side effect: overwritten memory into \texttt{s}
\end{itemize}

\begin{verbatim}
void format_matrix_into(Matrix_double *m, char *s) {
  if (m->rows == 0)
    strcpy(s, "empty");

  for (size_t y = 0; y < m->rows; ++y) {
    char row_s[256];
    strcpy(row_s, "");

    format_vector_into(m->data[y], row_s);
    strcat(s, row_s);
  }
  strcat(s, "\n");
}
\end{verbatim}
\subsection{Root Finding Methods}
\label{sec:org81f315b}
\subsubsection{\texttt{find\_ivt\_range}}
\label{sec:orgc1dde4d}
\begin{itemize}
\item Author: Elizabeth Hunt
\item Name: \texttt{find\_ivt\_range}
\item Location: \texttt{src/roots.c}
\item Input: a pointer to a oneary function taking a double and producing a double, the beginning point
in \(R\) to search for a range, a \texttt{delta} step that is taken, and a \texttt{max\_steps} number of maximum
iterations to perform.
\item Output: a pair of \texttt{double}'s in an \texttt{Array\_double} representing a closed closed interval \texttt{[beginning, end]}
\end{itemize}

\begin{verbatim}
// f is well defined at start_x + delta*n for all n on the integer range [0,
// max_iterations]
Array_double *find_ivt_range(double (*f)(double), double start_x, double delta,
                             size_t max_iterations) {
  double a = start_x;

  while (f(a) * f(a + delta) >= 0 && max_iterations > 0) {
    max_iterations--;
    a += delta;
  }

  double end = a + delta;
  double begin = a - delta;

  if (max_iterations == 0 && f(begin) * f(end) >= 0)
    return NULL;
  return InitArray(double, {begin, end});
}
\end{verbatim}
\subsubsection{\texttt{bisect\_find\_root}}
\label{sec:orgb42a836}
\begin{itemize}
\item Author: Elizabeth Hunt
\item Name(s): \texttt{bisect\_find\_root}
\item Input: a one-ary function taking a double and producing a double, a closed interval represented
by \texttt{a} and \texttt{b}: \texttt{[a, b]}, a \texttt{tolerance} at which we return the estimated root once \(b-a < \text{tolerance}\), and a
\texttt{max\_iterations} to break us out of a loop if we can never reach the \texttt{tolerance}.
\item Output: a vector of size of 3 \texttt{double}'s representing first the .
\item Description: recursively uses binary search to split the interval until we reach \texttt{tolerance}. We
also assume the function \texttt{f} is continuous on \texttt{[a, b]}.
\end{itemize}

\begin{verbatim}
// f is continuous on [a, b]
Array_double *bisect_find_root(double (*f)(double), double a, double b,
                               double tolerance, size_t max_iterations) {
  assert(a <= b);
  // guarantee there's a root somewhere between a and b by IVT
  assert(f(a) * f(b) < 0);

  double c = (1.0 / 2) * (a + b);
  if (b - a < tolerance || max_iterations == 0)
    return InitArray(double, a, b, c);

  if (f(a) * f(c) < 0)
    return bisect_find_root(f, a, c, tolerance, max_iterations - 1);
  return bisect_find_root(f, c, b, tolerance, max_iterations - 1);
}
\end{verbatim}
\subsubsection{\texttt{bisect\_find\_root\_with\_error\_assumption}}
\label{sec:org762134e}
\begin{itemize}
\item Author: Elizabeth Hunt
\item Name: \texttt{bisect\_find\_root\_with\_error\_assumption}
\item Input: a one-ary function taking a double and producing a double, a closed interval represented
by \texttt{a} and \texttt{b}: \texttt{[a, b]}, and a \texttt{tolerance} equivalent to the above definition in \texttt{bisect\_find\_root}
\item Output: a \texttt{double} representing the estimated root
\item Description: using the bisection method we know that \(e_k \le (\frac{1}{2})^k (b_0 - a_0)\). So we can
calculate \(k\) at the worst possible case (that the error is exactly the tolerance) to be
\(\frac{log(tolerance) - log(b_0 - a_0)}{log(\frac{1}{2})}\). We pass this value into the \texttt{max\_iterations}
of \texttt{bisect\_find\_root} as above.
\end{itemize}
\begin{verbatim}
double bisect_find_root_with_error_assumption(double (*f)(double), double a,
                                              double b, double tolerance) {
  assert(a <= b);

  uint64_t max_iterations =
      (uint64_t)ceil((log(tolerance) - log(b - a)) / log(1 / 2.0));

  Array_double *a_b_root = bisect_find_root(f, a, b, tolerance, max_iterations);
  double root = a_b_root->data[2];
  free_vector(a_b_root);

  return root;
}
\end{verbatim}
\subsubsection{\texttt{fixed\_point\_iteration\_method}}
\label{sec:org9f210ad}
\begin{itemize}
\item Author: Elizabeth Hunt
\item Name: \texttt{fixed\_point\_iteration\_method}
\item Location: \texttt{src/roots.c}
\item Input: a pointer to a oneary function \(f\) taking a double and producing a double of which we are
trying to find a root, a guess \(x_0\), and a function \(g\) of the same signature of \(f\) at which we
"step" our guesses according to the fixed point iteration method: \(x_k = g(x_{k-1})\). Additionally, a
\texttt{max\_iterations} representing the maximum number of "steps" to take before arriving at our
approximation and a \texttt{tolerance} to return our root if it becomes within [0 - tolerance, 0 + tolerance].
\item Assumptions: \(g(x)\) must be a function such that at the point \(x^*\) (the found root) the derivative
\(|g'(x^*)| \lt 1\)
\item Output: a double representing the found approximate root \(\approx x^*\).
\end{itemize}

\begin{verbatim}
double fixed_point_iteration_method(double (*f)(double), double (*g)(double),
                                    double x_0, double tolerance,
                                    size_t max_iterations) {
  if (max_iterations <= 0)
    return x_0;

  double root = g(x_0);
  if (tolerance >= fabs(f(root)))
    return root;

  return fixed_point_iteration_method(f, g, root, tolerance,
                                      max_iterations - 1);
}
\end{verbatim}
\subsubsection{\texttt{fixed\_point\_newton\_method}}
\label{sec:orgedecc45}
\begin{itemize}
\item Author: Elizabeth Hunt
\item Name: \texttt{fixed\_point\_newton\_method}
\item Location: \texttt{src/roots.c}
\item Input: a pointer to a oneary function \(f\) taking a double and producing a double of which we are
trying to find a root and another pointer to a function fprime of the same signature, a guess \(x_0\),
and a \texttt{max\_iterations} and \texttt{tolerance} as defined in the above method are required inputs.
\item Description: continually computes elements in the sequence \(x_n = x_{n-1} - \frac{f(x_{n-1})}{f'p(x_{n-1})}\)
\item Output: a double representing the found approximate root \(\approx x^*\) recursively applied to the sequence
given
\end{itemize}
\begin{verbatim}
double fixed_point_newton_method(double (*f)(double), double (*fprime)(double),
                                 double x_0, double tolerance,
                                 size_t max_iterations) {
  if (max_iterations <= 0)
    return x_0;

  double root = x_0 - f(x_0) / fprime(x_0);
  if (tolerance >= fabs(f(root)))
    return root;

  return fixed_point_newton_method(f, fprime, root, tolerance,
                                   max_iterations - 1);
}
\end{verbatim}
\subsubsection{\texttt{fixed\_point\_secant\_method}}
\label{sec:org63bcbe2}
\begin{itemize}
\item Author: Elizabeth Hunt
\item Name: \texttt{fixed\_point\_secant\_method}
\item Location: \texttt{src/roots.c}
\item Input: a pointer to a oneary function \(f\) taking a double and producing a double of which we are
trying to find a root, a guess \(x_0\), and a \(\delta\) of our first guess at which we draw the first
secant line according to the sequence \(x_n = x_{n-1} - f(x_{n-1}) \frac{x_{n-1} - x_{n-2}}{f(x_{n-1}) - f(x_{n-2})}\) which
thus simplifies to \(x_1 = (x_0 + \delta) - f(x_0 + \delta) \frac{(x_0 + \delta) - x_0}{f(x_0 + \delta) - f(x_0)} = (x_0 + \delta) - f(x_0 + \delta) \frac{\delta}{f(x_0 + \delta) - f(x_0)}\).
Additionally, a \texttt{max\_iterations} and \texttt{tolerance} as defined in the above method are required
inputs.
\item Output: a double representing the found approximate root \(\approx x^*\) recursively applied to the sequence.
\end{itemize}
\begin{verbatim}
double fixed_point_secant_method(double (*f)(double), double x_0, double delta,
                                 double tolerance, size_t max_iterations) {
  if (max_iterations <= 0)
    return x_0;

  double x_1 = x_0 + delta;
  double root = x_1 - f(x_1) * (delta / (f(x_1) - f(x_0)));

  if (tolerance >= fabs(f(root)))
    return root;

  double new_delta = root - x_1;
  return fixed_point_secant_method(f, x_1, new_delta, tolerance,
                                   max_iterations);
}
\end{verbatim}
\subsubsection{\texttt{fixed\_point\_secant\_bisection\_method}}
\label{sec:org72d3074}
\begin{itemize}
\item Author: Elizabeth Hunt
\item Name: \texttt{fixed\_point\_secant\_method}
\item Location: \texttt{src/roots.c}
\item Input: a pointer to a oneary function \(f\) taking a double and producing a double of which we are
trying to find a root, a guess \(x_0\), and a \(\delta\) of which we define our first interval \([x_0, x_0 + \delta]\).
Then, we perform a single iteration of the \texttt{fixed\_point\_secant\_method} on this interval; if it
produces a root outside, we refresh the interval and root respectively with the given
\texttt{bisect\_find\_root} method. Additionally, a \texttt{max\_iterations} and \texttt{tolerance} as defined in the above method are required
inputs.
\item Output: a double representing the found approximate root \(\approx x^*\) continually applied with the
constraints defined.
\end{itemize}

\begin{verbatim}
double fixed_point_secant_bisection_method(double (*f)(double), double x_0,
                                           double delta, double tolerance,
                                           size_t max_iterations) {
  double begin = x_0;
  double end = x_0 + delta;
  double root = x_0;

  while (tolerance < fabs(f(root)) && max_iterations > 0) {
    max_iterations--;

    double secant_root =
        fixed_point_secant_method(f, begin, end - begin, tolerance, 1);

    if (secant_root < begin || secant_root > end) {
      Array_double *range_root = bisect_find_root(f, begin, end, tolerance, 1);

      begin = range_root->data[0];
      end = range_root->data[1];
      root = range_root->data[2];

      free_vector(range_root);
      continue;
    }

    root = secant_root;
    // the root exists in [begin, secant_root]
    if (f(root) * f(begin) < 0)
      end = secant_root;
    else
      begin = secant_root;
  }

  return root;
}
\end{verbatim}
\subsection{Linear Routines}
\label{sec:org04f3e56}
\subsubsection{\texttt{least\_squares\_lin\_reg}}
\label{sec:orgbd48d8e}
\begin{itemize}
\item Author: Elizabeth Hunt
\item Name: \texttt{least\_squares\_lin\_reg}
\item Location: \texttt{src/lin.c}
\item Input: two pointers to \texttt{Array\_double}'s whose entries correspond two ordered
pairs in R\textsuperscript{2}
\item Output: a linear model best representing the ordered pairs via least squares
regression
\end{itemize}

\begin{verbatim}
Line *least_squares_lin_reg(Array_double *x, Array_double *y) {
  assert(x->size == y->size);

  uint64_t n = x->size;
  double sum_x = sum_v(x);
  double sum_y = sum_v(y);
  double sum_xy = v_dot_v(x, y);
  double sum_xx = v_dot_v(x, x);
  double denom = ((n * sum_xx) - (sum_x * sum_x));

  Line *line = malloc(sizeof(Line));
  line->m = ((sum_xy * n) - (sum_x * sum_y)) / denom;
  line->a = ((sum_y * sum_xx) - (sum_x * sum_xy)) / denom;

  return line;
}
\end{verbatim}
\subsection{Appendix / Miscellaneous}
\label{sec:orgf6b30a5}
\subsubsection{Data Types}
\label{sec:orgd382789}
\begin{enumerate}
\item \texttt{Line}
\label{sec:orgab590b9}
\begin{itemize}
\item Author: Elizabeth Hunt
\item Location: \texttt{inc/types.h}
\end{itemize}

\begin{verbatim}
typedef struct Line {
  double m;
  double a;
} Line;
\end{verbatim}
\item The \texttt{Array\_<type>} and \texttt{Matrix\_<type>}
\label{sec:org5be3024}
\begin{itemize}
\item Author: Elizabeth Hunt
\item Location: \texttt{inc/types.h}
\end{itemize}

We define two Pre processor Macros \texttt{DEFINE\_ARRAY} and \texttt{DEFINE\_MATRIX} that take
as input a type, and construct a struct definition for the given type for
convenient access to the vector or matrices dimensions.

Such that \texttt{DEFINE\_ARRAY(int)} would expand to:

\begin{verbatim}
typedef struct {
  int* data;
  size_t size;
} Array_int
\end{verbatim}

And \texttt{DEFINE\_MATRIX(int)} would expand a to \texttt{Matrix\_int}; containing a pointer to
a collection of pointers of \texttt{Array\_int}'s and its dimensions.

\begin{verbatim}
typedef struct {
  Array_int **data;
  size_t cols;
  size_t rows;
} Matrix_int
\end{verbatim}
\end{enumerate}
\subsubsection{Macros}
\label{sec:org20a391c}
\begin{enumerate}
\item \texttt{c\_max} and \texttt{c\_min}
\label{sec:orgfc6117a}
\begin{itemize}
\item Author: Elizabeth Hunt
\item Location: \texttt{inc/macros.h}
\item Input: two structures that define an order measure
\item Output: either the larger or smaller of the two depending on the measure
\end{itemize}

\begin{verbatim}
#define c_max(x, y) (((x) >= (y)) ? (x) : (y))
#define c_min(x, y) (((x) <= (y)) ? (x) : (y))
\end{verbatim}
\item \texttt{InitArray}
\label{sec:org472f039}
\begin{itemize}
\item Author: Elizabeth Hunt
\item Location: \texttt{inc/macros.h}
\item Input: a type and array of values to initialze an array with such type
\item Output: a new \texttt{Array\_type} with the size of the given array and its data
\end{itemize}

\begin{verbatim}
#define InitArray(TYPE, ...)                            \
  ({                                                    \
    TYPE temp[] = __VA_ARGS__;                          \
    Array_##TYPE *arr = malloc(sizeof(Array_##TYPE));   \
    arr->size = sizeof(temp) / sizeof(temp[0]);         \
    arr->data = malloc(arr->size * sizeof(TYPE));       \
    memcpy(arr->data, temp, arr->size * sizeof(TYPE));  \
    arr;                                                \
  })
\end{verbatim}
\item \texttt{InitArrayWithSize}
\label{sec:orgbe950b8}
\begin{itemize}
\item Author: Elizabeth Hunt
\item Location: \texttt{inc/macros.h}
\item Input: a type, a size, and initial value
\item Output: a new \texttt{Array\_type} with the given size filled with the initial value
\end{itemize}

\begin{verbatim}
#define InitArrayWithSize(TYPE, SIZE, INIT_VALUE)      \
  ({                                                   \
    Array_##TYPE *arr = malloc(sizeof(Array_##TYPE));  \
    arr->size = SIZE;                                  \
    arr->data = malloc(arr->size * sizeof(TYPE));      \
    for (size_t i = 0; i < arr->size; i++)             \
      arr->data[i] = INIT_VALUE;                       \
    arr;                                               \
  })
\end{verbatim}
\item \texttt{InitMatrixWithSize}
\label{sec:org5965f3b}
\begin{itemize}
\item Author: Elizabeth Hunt
\item Location: \texttt{inc/macros.h}
\item Input: a type, number of rows, columns, and initial value
\item Output: a new \texttt{Matrix\_type} of size \texttt{rows x columns} filled with the initial
value
\end{itemize}

\begin{verbatim}
#define InitMatrixWithSize(TYPE, ROWS, COLS, INIT_VALUE)                       \
  ({                                                                           \
    Matrix_##TYPE *matrix = malloc(sizeof(Matrix_##TYPE));                     \
    matrix->rows = ROWS;                                                       \
    matrix->cols = COLS;                                                       \
    matrix->data = malloc(matrix->rows * sizeof(Array_##TYPE *));              \
    for (size_t y = 0; y < matrix->rows; y++)                                  \
      matrix->data[y] = InitArrayWithSize(TYPE, COLS, INIT_VALUE);             \
    matrix;                                                                    \
  })
\end{verbatim}
\end{enumerate}
\end{document}
