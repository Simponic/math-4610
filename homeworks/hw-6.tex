% Created 2023-11-11 Sat 13:13
% Intended LaTeX compiler: pdflatex
\documentclass[11pt]{article}
\usepackage[utf8]{inputenc}
\usepackage[T1]{fontenc}
\usepackage{graphicx}
\usepackage{longtable}
\usepackage{wrapfig}
\usepackage{rotating}
\usepackage[normalem]{ulem}
\usepackage{amsmath}
\usepackage{amssymb}
\usepackage{capt-of}
\usepackage{hyperref}
\notindent \notag  \usepackage{amsmath} \usepackage[a4paper,margin=1in,portrait]{geometry}
\author{Elizabeth Hunt}
\date{\today}
\title{Homework 6}
\hypersetup{
 pdfauthor={Elizabeth Hunt},
 pdftitle={Homework 6},
 pdfkeywords={},
 pdfsubject={},
 pdfcreator={Emacs 29.1 (Org mode 9.7-pre)}, 
 pdflang={English}}
\begin{document}

\maketitle
\setlength\parindent{0pt}
\section{Question One}
\label{sec:org206b859}

For \(g(x) = x + f(x)\) then we know \(g'(x) = 1 + 2x - 5\) and thus \(|g'(x)| \lt 1\) is only true
on the interval \((1.5, 2.5)\), and for \(g(x) = x - f(x)\) then we know \(g'(x) = 1 - (2x - 5)\)
and thus \(|g'(x)| < 1\) is only true on the interval \((2.5, 3.5)\).

Because we know the roots of \(f\) are \(2, 3\) (\(f(x) = (x-2)(x-3)\)) then we can only be
certain that \(g(x) = x + f(x)\) will converge to the root \(2\) if we pick an initial
guess between \((1.5, 2.5)\), and likewise for \(g(x) = x - f(x)\), \(3\):

\begin{verbatim}
// tests/roots.t.c
UTEST(root, fixed_point_iteration_method) {
  // x^2 - 5x + 6 = (x - 3)(x - 2)
  double expect_x1 = 3.0;
  double expect_x2 = 2.0;

  double tolerance = 0.001;
  uint64_t max_iterations = 10;

  double x_0 = 1.55; // 1.5 < 1.55 < 2.5
  // g1(x) = x + f(x)
  double root1 =
      fixed_point_iteration_method(&f2, &g1, x_0, tolerance, max_iterations);
  EXPECT_NEAR(root1, expect_x2, tolerance);

  // g2(x) = x - f(x)
  x_0 = 3.4; // 2.5 < 3.4 < 3.5
  double root2 =
      fixed_point_iteration_method(&f2, &g2, x_0, tolerance, max_iterations);
  EXPECT_NEAR(root2, expect_x1, tolerance);
}
\end{verbatim}

And by this method passing in \texttt{tests/roots.t.c} we know they converged within \texttt{tolerance} before
10 iterations.
\section{Question Two}
\label{sec:orga0f5b42}

Yes, we showed that for \(\epsilon = 1\) in Question One, we can converge upon a root in the range \((2.5, 3.5)\), and
when \(\epsilon = -1\) we can converge upon a root in the range \((1.5, 2.5)\).

See the above unit tests in Question One for each \(\epsilon\).
\section{Question Three}
\label{sec:org19aa326}

See \texttt{test/roots.t.c -> UTEST(root, bisection\_with\_error\_assumption)}
and the software manual entry \texttt{bisect\_find\_root\_with\_error\_assumption}.
\section{Question Four}
\label{sec:org722aa6a}

See \texttt{test/roots.t.c -> UTEST(root, fixed\_point\_newton\_method)}
and the software manual entry \texttt{fixed\_point\_newton\_method}.
\section{Question Five}
\label{sec:org587ee52}

See \texttt{test/roots.t.c -> UTEST(root, fixed\_point\_secant\_method)}
and the software manual entry \texttt{fixed\_point\_secant\_method}.
\section{Question Six}
\label{sec:org79bf754}

See \texttt{test/roots.t.c -> UTEST(root, fixed\_point\_bisection\_secant\_method)}
and the software manual entry \texttt{fixed\_point\_bisection\_secant\_method}.
\section{Question Seven}
\label{sec:org4cb47e5}

The existance of \texttt{test/roots.t.c}'s compilation into \texttt{dist/lizfcm.test} via \texttt{make}
shows that the compiled \texttt{lizfcm.a} contains the root methods mentioned; a user
could link the library and use them, as we do in Question Eight.
\section{Question Eight}
\label{sec:org4a8160d}

The given ODE \(\frac{dP}{dt} = \alpha P - \beta P\) has a trivial solution by separation:

\begin{equation*}
P(t) = C e^{t(\alpha - \beta)}
\end{equation*}

And

\begin{equation*}
P_0 = P(0) = C e^0 = C
\end{equation*}

So \(P(t) = P_0 e^{t(\alpha - \beta)}\).

We're trying to find \(t\) such that \(P(t) = P_\infty\), thus we're finding roots of \(P(t) - P_\infty\).

The following code (in \texttt{homeworks/hw\_6\_p\_8.c}) produces this output:

\begin{verbatim}
$ gcc -I../inc/ -Wall hw_6_p_8.c ../lib/lizfcm.a -lm -o hw_6_p_8 && ./hw_6_p_8

a ~ 27.303411; P(27.303411) - P_infty = -0.000000
b ~ 40.957816; P(40.957816) - P_infty = -0.000000
c ~ 40.588827; P(40.588827) - P_infty = -0.000000
d ~ 483.611967; P(483.611967) - P_infty = -0.000000
e ~ 4.894274; P(4.894274) - P_infty = -0.000000

\end{verbatim}

\begin{verbatim}
// compile & test w/
//  \--> gcc -I../inc/ -Wall hw_6_p_8.c ../lib/lizfcm.a -lm -o hw_6_p_8
//  \--> ./hw_6_p_8

#include "lizfcm.h"
#include <math.h>
#include <stdio.h>

double a(double t) {
  double alpha = 0.1;
  double beta = 0.001;
  double p_0 = 2;
  double p_infty = 29.85;

  return p_0 * exp(t * (alpha - beta)) - p_infty;
}

double b(double t) {
  double alpha = 0.1;
  double beta = 0.001;
  double p_0 = 2;
  double p_infty = 115.35;

  return p_0 * exp(t * (alpha - beta)) - p_infty;
}

double c(double t) {
  double alpha = 0.1;
  double beta = 0.0001;
  double p_0 = 2;
  double p_infty = 115.35;

  return p_0 * exp(t * (alpha - beta)) - p_infty;
}

double d(double t) {
  double alpha = 0.01;
  double beta = 0.001;
  double p_0 = 2;
  double p_infty = 155.346;

  return p_0 * exp(t * (alpha - beta)) - p_infty;
}

double e(double t) {
  double alpha = 0.1;
  double beta = 0.01;
  double p_0 = 100;
  double p_infty = 155.346;

  return p_0 * exp(t * (alpha - beta)) - p_infty;
}

int main() {
  uint64_t max_iterations = 1000;
  double tolerance = 0.0000001;

  Array_double *ivt_range = find_ivt_range(&a, -5.0, 3.0, 1000);
  double approx_a = fixed_point_secant_bisection_method(
      &a, ivt_range->data[0], ivt_range->data[1], tolerance, max_iterations);

  free_vector(ivt_range);
  ivt_range = find_ivt_range(&b, -5.0, 3.0, 1000);
  double approx_b = fixed_point_secant_bisection_method(
      &b, ivt_range->data[0], ivt_range->data[1], tolerance, max_iterations);

  free_vector(ivt_range);
  ivt_range = find_ivt_range(&c, -5.0, 3.0, 1000);
  double approx_c = fixed_point_secant_bisection_method(
      &c, ivt_range->data[0], ivt_range->data[1], tolerance, max_iterations);

  free_vector(ivt_range);
  ivt_range = find_ivt_range(&d, -5.0, 3.0, 1000);
  double approx_d = fixed_point_secant_bisection_method(
      &d, ivt_range->data[0], ivt_range->data[1], tolerance, max_iterations);

  free_vector(ivt_range);
  ivt_range = find_ivt_range(&e, -5.0, 3.0, 1000);
  double approx_e = fixed_point_secant_bisection_method(
      &e, ivt_range->data[0], ivt_range->data[1], tolerance, max_iterations);

  printf("a ~ %f; P(%f) = %f\n", approx_a, approx_a, a(approx_a));
  printf("b ~ %f; P(%f) = %f\n", approx_b, approx_b, b(approx_b));
  printf("c ~ %f; P(%f) = %f\n", approx_c, approx_c, c(approx_c));
  printf("d ~ %f; P(%f) = %f\n", approx_d, approx_d, d(approx_d));
  printf("e ~ %f; P(%f) = %f\n", approx_e, approx_e, e(approx_e));

  return 0;
}
\end{verbatim}
\end{document}
