% Created 2023-11-27 Mon 15:13
% Intended LaTeX compiler: pdflatex
\documentclass[11pt]{article}
\usepackage[utf8]{inputenc}
\usepackage[T1]{fontenc}
\usepackage{graphicx}
\usepackage{longtable}
\usepackage{wrapfig}
\usepackage{rotating}
\usepackage[normalem]{ulem}
\usepackage{amsmath}
\usepackage{amssymb}
\usepackage{capt-of}
\usepackage{hyperref}
\notindent \notag  \usepackage{amsmath} \usepackage[a4paper,margin=1in,portrait]{geometry}
\author{Elizabeth Hunt}
\date{\today}
\title{Homework 7}
\hypersetup{
 pdfauthor={Elizabeth Hunt},
 pdftitle={Homework 7},
 pdfkeywords={},
 pdfsubject={},
 pdfcreator={Emacs 29.1 (Org mode 9.7-pre)}, 
 pdflang={English}}
\begin{document}

\maketitle
\setlength\parindent{0pt}
\section{Question One}
\label{sec:org8ef0ee6}
See \texttt{UTEST(eigen, dominant\_eigenvalue)} in \texttt{test/eigen.t.c} and the entry
\texttt{Eigen-Adjacent -> dominant\_eigenvalue} in the LIZFCM API documentation.
\section{Question Two}
\label{sec:orgbdba5c1}
See \texttt{UTEST(eigen, leslie\_matrix\_dominant\_eigenvalue)} in \texttt{test/eigen.t.c}
and the entry \texttt{Eigen-Adjacent -> leslie\_matrix} in the LIZFCM API
documentation.
\section{Question Three}
\label{sec:org19b04f4}
See \texttt{UTEST(eigen, least\_dominant\_eigenvalue)} in \texttt{test/eigen.t.c} which
finds the least dominant eigenvalue on the matrix:

\begin{bmatrix}
2 & 2 & 4 \\
1 & 4 & 7 \\
0 & 2 & 6 
\end{bmatrix}

which has eigenvalues: \(5 + \sqrt{17}, 2, 5 - \sqrt{17}\) and should thus produce \(5 - \sqrt{17}\).

See also the entry \texttt{Eigen-Adjacent -> least\_dominant\_eigenvalue} in the LIZFCM API
documentation.
\section{Question Four}
\label{sec:orgc58d42d}
See \texttt{UTEST(eigen, shifted\_eigenvalue)} in \texttt{test/eigen.t.c} which
finds the least dominant eigenvalue on the matrix:

\begin{bmatrix}
2 & 2 & 4 \\
1 & 4 & 7 \\
0 & 2 & 6 
\end{bmatrix}

which has eigenvalues: \(5 + \sqrt{17}, 2, 5 - \sqrt{17}\) and should thus produce \(2.0\).

With the initial guess: \([0.5, 1.0, 0.75]\).

See also the entry \texttt{Eigen-Adjacent -> shift\_inverse\_power\_eigenvalue} in the LIZFCM API
documentation.
\section{Question Five}
\label{sec:orga369221}
See \texttt{UTEST(eigen, partition\_find\_eigenvalues)} in \texttt{test/eigen.t.c} which
finds the eigenvalues in a partition of 10 on the matrix:

\begin{bmatrix}
2 & 2 & 4 \\
1 & 4 & 7 \\
0 & 2 & 6 
\end{bmatrix}

which has eigenvalues: \(5 + \sqrt{17}, 2, 5 - \sqrt{17}\), and should produce all three from
the partitions when given the guesses \([0.5, 1.0, 0.75]\) from the questions above.

See also the entry \texttt{Eigen-Adjacent -> partition\_find\_eigenvalues} in the LIZFCM API
documentation.
\section{Question Six}
\label{sec:orgadc3078}
Consider we have the results of two methods developed in this homework: \texttt{least\_dominant\_eigenvalue}, and \texttt{dominant\_eigenvalue}
into \texttt{lambda\_0}, \texttt{lambda\_n}, respectively. Also assume that we have the method implemented as we've introduced,
\texttt{shift\_inverse\_power\_eigenvalue}.

Then, we begin at the midpoint of \texttt{lambda\_0} and \texttt{lambda\_n}, and compute the
\texttt{new\_lambda = shift\_inverse\_power\_eigenvalue}
with a shift at the midpoint, and some given initial guess.

\begin{enumerate}
\item If the result is equal (or within some tolerance) to \texttt{lambda\_n} then the closest eigenvalue to the midpoint
is still the dominant eigenvalue, and thus the next most dominant will be on the left. Set \texttt{lambda\_n}
to the midpoint and reiterate.
\item If the result is greater or equal to \texttt{lambda\_0} we know an eigenvalue of greater or equal magnitude
exists on the right. So, we set \texttt{lambda\_0} to this eigenvalue associated with the midpoint, and
re-iterate.
\item Continue re-iterating until we hit some given maximum number of iterations. Finally we will return
\texttt{new\_lambda}.
\end{enumerate}
\end{document}
