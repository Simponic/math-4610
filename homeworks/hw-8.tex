% Created 2023-12-09 Sat 22:06
% Intended LaTeX compiler: pdflatex
\documentclass[11pt]{article}
\usepackage[utf8]{inputenc}
\usepackage[T1]{fontenc}
\usepackage{graphicx}
\usepackage{longtable}
\usepackage{wrapfig}
\usepackage{rotating}
\usepackage[normalem]{ulem}
\usepackage{amsmath}
\usepackage{amssymb}
\usepackage{capt-of}
\usepackage{hyperref}
\notindent \notag  \usepackage{amsmath} \usepackage[a4paper,margin=1in,portrait]{geometry}
\author{Elizabeth Hunt}
\date{\today}
\title{Homework 8}
\hypersetup{
 pdfauthor={Elizabeth Hunt},
 pdftitle={Homework 8},
 pdfkeywords={},
 pdfsubject={},
 pdfcreator={Emacs 28.2 (Org mode 9.7-pre)}, 
 pdflang={English}}
\begin{document}

\maketitle
\setlength\parindent{0pt}

\section{Question One}
\label{sec:org800c743}
See \texttt{UTEST(jacobi, solve\_jacobi)} in \texttt{test/jacobi.t.c} and the entry
\texttt{Jacobi / Gauss-Siedel -> solve\_jacobi} in the LIZFCM API documentation.
\section{Question Two}
\label{sec:org6121bef}
We cannot just perform the Jacobi algorithm on a Leslie matrix since
it is obviously not diagonally dominant - which is a requirement. It is
certainly not always the case, but, if a Leslie matrix \(L\) is invertible, we can
first perform gaussian elimination on \(L\) augmented with \(n_{k+1}\)
to obtain \(n_k\) with the Jacobi method. See \texttt{UTEST(jacobi, leslie\_solve)}
in \texttt{test/jacobi.t.c} for an example wherein this method is tested on a Leslie
matrix to recompute a given initial population distribution.

In terms of accuracy, an LU factorization and back substitution approach will
always be as correct as possible within the limits of computation; it's a
direct solution method. It's simply the nature of the Jacobi algorithm being
a convergent solution that determines its accuracy.

LU factorization also performs in order \(O(n^3)\) runtime for an \(n \times n\)
matrix, whereas the Jacobi algorithm runs in order \(O(k n^2) = O(n^2)\) on average
but with the con that \(k\) is given by some function on both the convergence criteria and the number of
nonzero entries in the matrix - which might end up worse in some cases than the LU decomp approach.

\section{Question Three}
\label{sec:org11282e6}
See \texttt{UTEST(jacobi, gauss\_siedel\_solve)} in \texttt{test/jacobi.t.c} which runs the same
unit test as \texttt{UTEST(jacobi, solve\_jacobi)} but using the
\texttt{Jacobi / Gauss-Siedel -> gauss\_siedel\_solve} method as documented in the LIZFCM API reference.

\section{Question Four, Five}
\label{sec:org22b52a9}
We produce the following operation counts (by hackily adding the operation count as the last element
to the solution vector) and errors - the sum of each vector elements' absolute value away from 1.0
using the proceeding patch and unit test.

\begin{center}
\begin{tabular}{rrrrrrr}
N & JAC opr & JAC err & GS opr & GS err & LU opr & LU err\\[0pt]
5 & 1622 & 0.001244 & 577 & 0.000098 & 430 & 0.000000\\[0pt]
6 & 2812 & 0.001205 & 775 & 0.000080 & 681 & 0.000000\\[0pt]
7 & 5396 & 0.001187 & 860 & 0.000178 & 1015 & 0.000000\\[0pt]
8 & 5618 & 0.001468 & 1255 & 0.000121 & 1444 & 0.000000\\[0pt]
9 & 7534 & 0.001638 & 1754 & 0.000091 & 1980 & 0.000000\\[0pt]
10 & 10342 & 0.001425 & 1847 & 0.000435 & 2635 & 0.000000\\[0pt]
11 & 12870 & 0.001595 & 2185 & 0.000368 & 3421 & 0.000000\\[0pt]
12 & 17511 & 0.001860 & 2912 & 0.000322 & 4350 & 0.000000\\[0pt]
13 & 16226 & 0.001631 & 3362 & 0.000270 & 5434 & 0.000000\\[0pt]
14 & 34333 & 0.001976 & 3844 & 0.000121 & 6685 & 0.000000\\[0pt]
15 & 38474 & 0.001922 & 4358 & 0.000311 & 8115 & 0.000000\\[0pt]
16 & 40405 & 0.002061 & 4904 & 0.000204 & 9736 & 0.000000\\[0pt]
17 & 58518 & 0.002125 & 5482 & 0.000311 & 11560 & 0.000000\\[0pt]
18 & 68079 & 0.002114 & 6092 & 0.000279 & 13599 & 0.000000\\[0pt]
19 & 95802 & 0.002159 & 6734 & 0.000335 & 15865 & 0.000000\\[0pt]
20 & 85696 & 0.002141 & 7408 & 0.000289 & 18370 & 0.000000\\[0pt]
21 & 89026 & 0.002316 & 8114 & 0.000393 & 21126 & 0.000000\\[0pt]
22 & 101537 & 0.002344 & 8852 & 0.000414 & 24145 & 0.000000\\[0pt]
23 & 148040 & 0.002323 & 9622 & 0.000230 & 27439 & 0.000000\\[0pt]
24 & 137605 & 0.002348 & 10424 & 0.000213 & 31020 & 0.000000\\[0pt]
25 & 169374 & 0.002409 & 11258 & 0.000894 & 34900 & 0.000000\\[0pt]
26 & 215166 & 0.002502 & 12124 & 0.000564 & 39091 & 0.000000\\[0pt]
27 & 175476 & 0.002616 & 13022 & 0.000535 & 43605 & 0.000000\\[0pt]
28 & 268454 & 0.002651 & 13952 & 0.000690 & 48454 & 0.000000\\[0pt]
29 & 267034 & 0.002697 & 14914 & 0.000675 & 53650 & 0.000000\\[0pt]
30 & 277193 & 0.002686 & 15908 & 0.000542 & 59205 & 0.000000\\[0pt]
31 & 336792 & 0.002736 & 16934 & 0.000390 & 65131 & 0.000000\\[0pt]
32 & 293958 & 0.002741 & 17992 & 0.000660 & 71440 & 0.000000\\[0pt]
33 & 323638 & 0.002893 & 19082 & 0.001072 & 78144 & 0.000000\\[0pt]
34 & 375104 & 0.003001 & 20204 & 0.001018 & 85255 & 0.000000\\[0pt]
35 & 436092 & 0.003004 & 21358 & 0.000912 & 92785 & 0.000000\\[0pt]
36 & 538143 & 0.003005 & 22544 & 0.000954 & 100746 & 0.000000\\[0pt]
37 & 511886 & 0.003029 & 23762 & 0.000462 & 109150 & 0.000000\\[0pt]
38 & 551332 & 0.003070 & 25012 & 0.000996 & 118009 & 0.000000\\[0pt]
39 & 592750 & 0.003110 & 26294 & 0.000989 & 127335 & 0.000000\\[0pt]
40 & 704208 & 0.003165 & 27608 & 0.000583 & 137140 & 0.000000\\[0pt]
\end{tabular}
\end{center}

\begin{verbatim}
diff --git a/src/matrix.c b/src/matrix.c
index 901a426..af5529f 100644
--- a/src/matrix.c
+++ b/src/matrix.c
@@ -144,20 +144,54 @@ Array_double *solve_matrix_lu_bsubst(Matrix_double *m, Array_double *b) {
   assert(b->size == m->rows);
   assert(m->rows == m->cols);

+  double opr = 0;
+
+  opr += b->size;
   Array_double *x = copy_vector(b);
+
+  size_t n = m->rows;
+  opr += n * n;     // (u copy)
+  opr += n * n;     // l_empty
+  opr += n * n + n; // copy + put_identity_diagonal
+  opr += n;         // pivot check
+  opr += m->cols;
+  for (size_t x = 0; x < m->cols; x++) {
+    opr += (m->rows - (x + 1));
+    for (size_t y = x + 1; y < m->rows; y++) {
+      opr += 1;
+      opr += 2;     // -factor
+      opr += 4 * n; // scale, add_v, free_vector
+      opr += 1;     // -factor
+    }
+  }
+  opr += n;
   Matrix_double **u_l = lu_decomp(m);
+
   Matrix_double *u = u_l[0];
   Matrix_double *l = u_l[1];

+  opr += n;
+  for (int64_t row = n - 1; row >= 0; row--) {
+    opr += 2 * (n - row);
+    opr += 1;
+  }
   Array_double *b_fsub = fsubst(l, b);
+
+  opr += n;
+  for (size_t x = 0; x < n; x++) {
+    opr += 2 * (x + 1);
+    opr += 1; // /= l->data[row]->data[row]
+  }
   x = bsubst(u, b_fsub);
-  free_vector(b_fsub);

+  free_vector(b_fsub);
   free_matrix(u);
   free_matrix(l);
   free(u_l);

-  return x;
+  Array_double *copy = add_element(x, opr);
+  free_vector(x);
+  return copy;
 }

 Matrix_double *gaussian_elimination(Matrix_double *m) {
@@ -231,18 +265,36 @@ Array_double *jacobi_solve(Matrix_double *m, Array_double *b,
   assert(b->size == m->cols);
   size_t iter = max_iterations;

+  double opr = 0;
+
+  opr += 2 * b->size; // to initialize two vectors with the same dim of b twice
   Array_double *x_k = InitArrayWithSize(double, b->size, 0.0);
   Array_double *x_k_1 =
       InitArrayWithSize(double, b->size, rand_from(0.1, 10.0));

+  // add since these wouldn't be accounter for after the loop
+  opr += 1; // iter decrement
+  opr +=
+      3 * x_k_1->size; // 1 to perform x_k_1, x_k and 2 to perform ||x_k_1||_2
   while ((--iter) > 0 && l2_distance(x_k_1, x_k) > l2_convergence_tolerance) {
+    opr += 1; // iter decrement
+    opr +=
+        3 * x_k_1->size; // 1 to perform x_k_1, x_k and 2 to perform ||x_k_1||_2
+
+    opr += m->rows; // row for add oprs
     for (size_t i = 0; i < m->rows; i++) {
       double delta = 0.0;
+
+      opr += m->cols;
       for (size_t j = 0; j < m->cols; j++) {
         if (i == j)
           continue;
+
+        opr += 1;
         delta += m->data[i]->data[j] * x_k->data[j];
       }
+
+      opr += 2;
       x_k_1->data[i] = (b->data[i] - delta) / m->data[i]->data[i];
     }

@@ -251,8 +303,9 @@ Array_double *jacobi_solve(Matrix_double *m, Array_double *b,
     x_k_1 = tmp;
   }

-  free_vector(x_k);
-  return x_k_1;
+  Array_double *copy = add_element(x_k_1, opr);
+  free_vector(x_k_1);
+  return copy;
 }

 Array_double *gauss_siedel_solve(Matrix_double *m, Array_double *b,
@@ -262,30 +315,48 @@ Array_double *gauss_siedel_solve(Matrix_double *m, Array_double *b,
   assert(b->size == m->cols);
   size_t iter = max_iterations;

+  double opr = 0;
+
+  opr += 2 * b->size; // to initialize two vectors with the same dim of b twice
   Array_double *x_k = InitArrayWithSize(double, b->size, 0.0);
   Array_double *x_k_1 =
       InitArrayWithSize(double, b->size, rand_from(0.1, 10.0));

   while ((--iter) > 0) {
+    opr += 1; // iter decrement
+
+    opr += x_k->size; // copy oprs
     for (size_t i = 0; i < x_k->size; i++)
       x_k->data[i] = x_k_1->data[i];

+    opr += m->rows; // row for add oprs
     for (size_t i = 0; i < m->rows; i++) {
       double delta = 0.0;
+
+      opr += m->cols;
       for (size_t j = 0; j < m->cols; j++) {
         if (i == j)
           continue;
+
+        opr += 1;
         delta += m->data[i]->data[j] * x_k_1->data[j];
       }
+
+      opr += 2;
       x_k_1->data[i] = (b->data[i] - delta) / m->data[i]->data[i];
     }

+    opr +=
+        3 * x_k_1->size; // 1 to perform x_k_1, x_k and 2 to perform ||x_k_1||_2
     if (l2_distance(x_k_1, x_k) <= l2_convergence_tolerance)
       break;
   }

   free_vector(x_k);
-  return x_k_1;
+
+  Array_double *copy = add_element(x_k_1, opr);
+  free_vector(x_k_1);
+  return copy;
 }
\end{verbatim}


And this unit test:
\begin{verbatim}
UTEST(hw_8, p4_5) {
  printf("| N | JAC opr | JAC err | GS opr | GS err | LU opr | LU err | \n");

  for (size_t i = 5; i < 100; i++) {
    Matrix_double *m = generate_ddm(i);
    double oprs[3] = {0.0, 0.0, 0.0};
    double errs[3] = {0.0, 0.0, 0.0};

    Array_double *b_1 = InitArrayWithSize(double, m->rows, 1.0);
    Array_double *b = m_dot_v(m, b_1);
    double tolerance = 0.001;
    size_t max_iter = 400;

    // JACOBI
    {
      Array_double *solution_with_opr_count =
          jacobi_solve(m, b, tolerance, max_iter);
      Array_double *solution = slice_element(solution_with_opr_count,
                                             solution_with_opr_count->size - 1);

      for (size_t i = 0; i < solution->size; i++)
        errs[0] += fabs(solution->data[i] - 1.0);

      oprs[0] =
          solution_with_opr_count->data[solution_with_opr_count->size - 1];

      free_vector(solution);
      free_vector(solution_with_opr_count);
    }

    // GAUSS-SIEDEL
    {
      Array_double *solution_with_opr_count =
          gauss_siedel_solve(m, b, tolerance, max_iter);
      Array_double *solution = slice_element(solution_with_opr_count,
                                             solution_with_opr_count->size - 1);

      for (size_t i = 0; i < solution->size; i++)
        errs[1] += fabs(solution->data[i] - 1.0);

      oprs[1] =
          solution_with_opr_count->data[solution_with_opr_count->size - 1];

      free_vector(solution);
      free_vector(solution_with_opr_count);
    }

    // LU-BSUBST
    {
      Array_double *solution_with_opr_count = solve_matrix_lu_bsubst(m, b);
      Array_double *solution = slice_element(solution_with_opr_count,
                                             solution_with_opr_count->size - 1);

      for (size_t i = 0; i < solution->size; i++)
        errs[2] += fabs(solution->data[i] - 1.0);

      oprs[2] =
          solution_with_opr_count->data[solution_with_opr_count->size - 1];

      free_vector(solution);
      free_vector(solution_with_opr_count);
    }
    free_matrix(m);
    free_vector(b_1);
    free_vector(b);

    printf("| %zu | %f | %f | %f | %f | %f | %f | \n", i, oprs[0], errs[0],
           oprs[1], errs[1], oprs[2], errs[2]);
  }
}
\end{verbatim}
\end{document}